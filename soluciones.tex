\documentclass[a4paper]{article}

\usepackage{pgfplots}
\usepackage{tikz}
\usepackage[utf8]{inputenc}
\usepackage[T1]{fontenc}
\usepackage{textcomp}
\usepackage[spanish]{babel}
\usepackage{amsmath, amssymb}


% figure support
\usepackage{import}
\usepackage{xifthen}
\pdfminorversion=7
\usepackage{pdfpages}
\usepackage{transparent}
\newcommand{\incfig}[1]{%
	\def\svgwidth{\columnwidth}
	\import{./figures/}{#1.pdf_tex}
}

\pdfsuppresswarningpagegroup=1
\title{Soluciones de las preguntas de teoría}
\author{Santiago Sanz Wuhl}

\begin{document}
\maketitle

\paragraph{Explique en qué consisten las hipótesis de medio continuo
(partícula fluida) y de equilibrio termodinámico local, indicando
los parámetros que loas caracterizan ¿Por qué la segunda es mucho
más restrictiva que la primera?} 

Al estudiar la mecánica de fluidos, el primer problema que uno se 
encuentra es que los fluidos son materia y la materia es discreta.
Para simplificar los cálculos, vamos a dividir el fluido en cuestión
en diferentes parcelas de volumen $\delta V$, de forma que 
a los parámetros que definen el sistema se les pueda asignar un valor
continuo en función de $\vec{x}$ y el tiempo, donde la parcela 
(partícula fluida) se ubica en las proximidades de $\vec{x}$. 

Esta parcela tiene que ser mucho mayor que la distancia media $d$ entre
las partículas que forman el medio, pero mucho menor que la longitud
característica del medio L (la longitud que hay que recorrer para 
poder apreciar variaciones en los parámetros que definen el sistema).

Con estas restricciones tenemos que $\delta V$ es lo suficientemente
grande como para que añadir una partícula no represente un cambio 
significativo en el parámetro a medir y por tanto éste no dependa
del tamaño de $\delta V$, pero lo suficientemente pequeño como para
que los parámetros sean constantes a lo largo de la partícula fluida.

La termodinámica estudia sistemas en equilibrio y por tanto todos los
parámetros que caracterizan un sistema serán constantes en el tiempo y
en el espacio. En la mecánica de fluidos los sistemas que se estudian 
son sistemas que no están en equilibrio, y por tanto pueden variar
tanto en el espacio como en el tiempo.

La hipótesis de equilibrio termodinámico local consiste en asumir que 
las partículas fluidas se encuentran en equilibrio termodinámico, lo
que nos permite usar resultados de la termodinámica y definir la 
presión y la temperatura.

Como las partículas transmiten momento y energía mediante colisiones,
ajustan su estado con las partículas vecinas. Por tanto si definimos
el recorrido libre medio $\lambda$ como la distancia que recorre una partícula
antes de chocar con otra partícula, si $\lambda \ll L$, entonces una
partícula chocará con su entorno muchas veces antes de cambiar a 
una región del espacio en la que las propiedades sean diferentes. 

De esta forma la condición de equilibrio termodinámico local es uqe $\lambda \ll L$.

Esta hipótesis es mucho más restrictiva que la hipótesis de medio 
continuo ya que $\lambda$ es mucho más grande que $d$.
\newpage
\paragraph{Indique por qué el movimiento en el entorno de un punto 
fluido está caracterizado por el tensor gradiente de velocidades.
Diga en qué dos tensores se descompone éste y lo que cada uno de ellos
representa. Por facilidad utilize coordenadas cartesianas.} 
Supongamos una línea fluida ubicada entre los puntos $\vec{x}$ y $\vec{x} + \vec{dx}$.
Como la velocidad no es uniforme, en primera aproximación valdrá 
$$\vec{v}(\vec{x}+\vec{dx}) = \vec{v}(\vec{x}) + \vec{dx} \nabla \vec{v}$$ en $\vec{x} + \vec{dx}$.
De esta forma, transcurrido un tiempo $dt$, los extremos de la línea
fluida se encontrarán en $\vec{x} + \vec{v}dt$, y $(\vec{x} + \vec{dx}) + (\vec{v} + \vec{dv})dt$.
No es difícil ver que el término $\vec{v}dt$ se corresponde a una traslación
mientras que el término $\vec{dv}dt$ se corresponde a una deformación.

Dado que $\vec{dv}= \vec{dx} \nabla \vec{v}$, podemos separar $\nabla \vec{v}$ 
en un tensor simétrico y en un tensor antisimétrico, de forma que 
\begin{align*}
	\nabla \vec{v} = \frac{1}{2} \left( \nabla \vec{v} + \nabla \vec{v} ^T \right)
	+ \frac{1}{2} \left(\nabla \vec{v} -\nabla \vec{v} ^T \right) =\overline{\overline {T}}_d +\overline{\overline {T}}_r
\end{align*}

\begin{align*}
	T_r = \frac{1}{2}\begin{bmatrix} 0 & \partial_1 v_2 - \partial_2 v_1& \partial_1 v_3 - \partial_3 v_1 \\
		\partial_2 v_1 - \partial_1 v_2 &
		0 &
		\partial_2 v_3 - \partial_3 v_2 \\
		\partial_3 v_1 - \partial_1 v_3&
		\partial_3 v_2 - \partial_2 v_3&
	0
	\end{bmatrix} 
\end{align*}
Y se puede observar que los elementos no diagonales son las componentes
de un vector $\vec{\omega} = \nabla \times \vec{v}  $ que es dos 
veces la vorticidad. 

Vemos así que si $\overline{\overline{T}}_d$ es nulo, el movimiento
se simplifica a un movimiento de sólido rígido, de rotación y de 
traslación.

$\overline{\overline{T}}_d$ representa por tanto la deformación. Como
la deformación no tiene por qué necesariamente ser en la dirección
de $\vec{dx}$, vemos cuánto se ha dilatado la línea fluida 
proyectando $\overline{\overline{T}}_d$ sobre  $\vec{dx}$. Si llamamos
$\hat{n}$ al vector unitario que rige la dirección de $\vec{dx}$, 
entonces la velocidad de deformación será 
\begin{align*}
	d \vec{v}_d = \overline{\overline{T}}_d\cdot \vec{dx}   = \overline{\overline{T}}_d\cdot \hat{n}  dx 
\end{align*}
y como ya hemos dicho, en general $\vec{dx}$ no apunta en la dirección
de la deformación, por lo que la velocidad de deformación en la 
dirección $\hat{n}$ es 
\begin{align*}
	\hat{n}\cdot \overline{\overline{T}}_d\cdot \hat{n}  dx
\end{align*}
\newpage

\paragraph{Deformación de un elemento cúbico de fluido} 

Supongamos un elemento cúbico de fluido con aristas $\vec{dl}$.
En un siguiente instante las aristas se encontrarán en 
\begin{align*}
\vec{dl} + \vec{dv}dt =  \vec{dl} + \vec{dx}\nabla \vec{v}dt 
\end{align*}
Y por tanto vemos que por ejemplo la arista que inicialmente 
se encontraba en 
\begin{align*}
	\vec{dl} = dl (1, 0, 0)
\end{align*}
se encuentra ahora en la posición 
\begin{align*}
	\vec{dl} = dl(1 + \partial_1 v_1 dt, \partial_1 v_2dt,  \partial_1 v_3dt)
\end{align*}

Si estudiamos ahora el plano $x_1x_2$ vemos que el ángulo que estos
dos ejes forman ha variado una cantidad  
\begin{align*}
\partial_1 v_2 + \partial_2 v_1
\end{align*}
Generalizando para las tres direcciones, vemos que los componentes no
diagonales del tensor unitario de velocidades de deformación son 
las componentes de la mitad de la velocidad de deformación angular.

Por otro lado, si consideramos que las direcciones son las direcciones
principales de deformación, el tensor será diagonal y las deformaciones
serán únicamente longitudinales. 

Finalmente, el nuevo volumen de nuestro cubo será 
\begin{align*}
dl 
\begin{vmatrix}
	1 + \partial_1 v_1dt &\partial_1 v_2dt &\partial_1 v_3dt \\
	\partial_2 v_1dt &1 + \partial_2 v_2dt &\partial_2 v_3dt \\
	\partial_3 v_1dt &\partial_3 v_2dt &1 + \partial_3 v_3dt \\
\end{vmatrix}
= dl^3 + (\partial_1 v_1 + \partial_2 v_2 + \partial_3 v_3) dt
\end{align*}
Y concluimos finalmente que $\nabla\cdot \vec{v}$ es la velocidad
de dilatación cúbica unitaria.

\newpage
\paragraph{Encuentre la ecuación de continuidad a partir del
balance de masa en un volumen fluido. A partir de la ecuación de 
continuidad encuentre la condición matemática que debe cumplir
el campo euleriano de velocidades de un fluido para que éste sea
incompresible.} 

En un volumen fluido la ecuación de balance de masa no es más que 
el teorema de transporte de Reynolds aplicado a la magnitud fluida 
$\rho$, o masa por unidad de volumen. 
\begin{align*}
	\frac{d}{dt}\int_{V_f} \rho dV = \frac{d}{dt}\int_{V_c} \rho dV + \oint_{\Sigma_c} \rho (\vec{v} - \vec{v_c})d \vec{s} = 0
\end{align*}
que se iguala a cero por el principio de conservación de la masa.

Si suponemos, por simplicidad, un volumen de control constante, 
entonces sabemos que $\vec{v}_c= 0$
Usando el teorema de Gauss transformamos la integral de superficie 
en una integral de volumen 
\begin{align*}
	\oint_{\Sigma_c} \rho \vec{v} d\vec{s} = 
	\int_{V_c} \nabla \cdot (\rho \vec{v})d \vec{s} 
\end{align*}
y finalmente obtenemos que 
\begin{align*}
	\frac{d}{dt}\int_{V_c} \rho dV +	\int_{V_c} \nabla \cdot (\rho \vec{v})d \vec{s}  = \int_{V_c}\left[  \frac{\partial \rho}{\partial t} + \nabla\cdot (\rho \vec{v}) \right]dV  = 0
\end{align*}

Como esta ecuación debe ser válida para cualquier $V_c$, se debe cumplir
que el integrando sea nulo, y llegamos así a la ecuación de continuidad
\begin{align*}
  \frac{\partial \rho}{\partial t} + \nabla\cdot (\rho \vec{v}) = 0
\end{align*}

Sabemos que para que un fluido sea incompresible tiene que cumplir que
$\rho$ es constante. Operando en la ecuación de continuidad
\begin{align*}
	\frac{\partial \rho}{\partial t} + \nabla \cdot (\rho \vec{v}) &= 
\rho\nabla\cdot  \vec{v} = 0
\end{align*}
Y por tanto concluimos que la condición que ha de cumplir es que $\nabla \cdot  \vec{v} = 0$

\newpage
\paragraph{Fuerzas de volumen y superficie sobre una partícula fluida.
El tensor de escuerzos; deducción y criterio de signos.} 
Las fuerzas que actúan sobre un fluido (o un sólido) pueden ser 
de volumen o de superficie. Las fuerzas de volumen son aquellas que 
tienen una longitud característica mucho mayor que $d$ y un radio
de acción del orden de $L$. Estas fuerzas son capaces de adentrarse
en el fluido y actuar sobre todas las partículas fluidas que lo 
conforman.

En la interfase entre dos superficies (reales o imaginarias) aparecen
fuerzas de origen molecular que decaen muy rápido, y no son 
apreciables a distancias mayores que $d$. Estas fuerzas aparecen
debido a la agitación térmica de las partículas que intercambian 
momento y energía con sus alrededores, dando así la aparición 
macroscópica de fuerzas superficiales. 

Estas fuerzas dependen de la orientación, y por tanto llamamos
$\vec{f}_n$ al esfuerzo en la dirección $\hat{n}$. Como $\hat{n}$ solo
depende de la orientación, tenemos 2 grados de libertad y por tanto
una doble infinitud de fuerzas a tener cuenta. En realidad el número
de dependencias es mucho más pequeño. 

$\vec{f}_n$ es el esfuerzo que ejerce el fluido situado en el lado hacia el que 
está dirigido $\hat{n}$ sobre el fluido situado en el lado contrario.

Sea el tetraedro de Cauchy de aristas $dx1dx2dx3$, y $\vec{f}_1$, $\vec{f}_2$ y $\vec{f}_3$ 
los esfuerzos aplicados a cada cara respectivamente (la cara $x_1$ 
es la perpendicular a $x_1$). Como las fuerzas másicas son proporcionales
a $dx^3$ (siendo $dx$ del orden de $dx1,dx2,dx3$), y las fuerzas
de superficie proporcionales a $dx^2$, podemos despreciar las fuerzas
másicas en la superficie reduciendo nuestro problema a un problema 
de equilibrio de esfuerzos, de forma que 
\begin{align*}
	\vec{f}_1 dA_1 + \vec{f}_2 dA_2 + \vec{f}_3 dA_3 = \vec{f}_n dA_n
\end{align*}
Si  $dA_n$ es la cara perpendicular a $\hat{n}$, entonces debe cumplirse que
\begin{align*}
	dA_i = n_i dA
\end{align*}
Siendo $\hat{n} = \left[ n1, n2, n3\right]$ y por tanto 
\begin{align*}
	\vec{f}_m =  n_1 \vec{f}_1 + n_2 \vec{f}_2+n_3 \vec{f}_3 = \hat{n}
	\begin{bmatrix}\tau_{11} &\tau_{12} &\tau_{13} \\\tau_{21} &\tau_{22} &\tau_{23} \\\tau_{31} &\tau_{32} &\tau_{33}   \end{bmatrix} = \hat{n} \cdot \overline{\overline{\tau}}
\end{align*}
De forma que $\overline{\overline{\tau}}$ es el llamado tensor de esfuerzos.


\newpage
\paragraph{¿Qué son fluidos newtonianos?¿Qué condición deben cumplir
los fluidos newtonianos para que el tensor de esfuerzos viscosos tome
la forma 
\begin{align*}
	\overline{\overline{\tau}}' = 2\mu \overline{\overline{T}}_d + (\mu_v - \frac{2}{3}\mu) \nabla\cdot \vec{v} \overline{\overline{I}}
\end{align*}
¿Qué son los coeficientes $\mu$ y $\mu_v$?} 

Los fluidos newtonianos son aquellos que cumplen que el tensor de 
esfuerzos viscosos es proporcional a las velocidades de deformación.
\begin{align*}
\tau'_{ij} = \alpha_{ijkl}\gamma_{kl}
\end{align*}

Para que el tensor de esfuerzos viscosos tome la forma dada se debe
cumplir que el fluido sea isótropo. De esa forma al expresar los 
tensores en la base principal de $\overline{\overline{T}}_d$ tenemos
que la proporcionalidad depende únicamente de 2 constantes. 

Los coeficientes $\mu$ y $\mu_v$ son los coeficientes de viscosidad
y de viscosidad cinemáticos, respectivamente. Se relacionan a través
del coeficiente segundo de viscosidad $\mu_v = \lambda + \frac{2}{3}\mu_v$

\newpage
\paragraph{Enuncie la ley de Navier Poisson e indique las condiciones
que debe cumplir un fluido para que se verifique dicha ley y las 
implicaciones de cada una de esas condiciones} 

Si el fluido se encuentra en reposo en algún sistema de referencia
la ley de Poisson dice que los esfuerzos son todos normales a la
superficie del fluido y no dependen de la direción
En cualquier sistema de referencia se puede escribir el tensor
de esfuerzos viscosos como
\begin{align*}
	\overline{\overline{\tau}}' = 2\mu \overline{\overline{T}}_d + (\mu_v - \frac{2}{3}\mu) \nabla\cdot \vec{v} \overline{\overline{I}}
\end{align*}

Para que esto se verifique el fluido ha de ser newtoniano e isótropo.
Un fluido newtoniano es aquel que el tensor de esfuerzos viscosos es
proporcional al tensor gradiente de velocidades, y al ser isótropo,
el tensor de velocidades unitarias de deformación es diagonal.


\newpage
\paragraph{Sea la ecuación 
\begin{align*}
	\rho \frac{D \vec{v}}{Dt} = - \nabla p + \nabla \cdot \overline{\overline{\tau}} + \rho \vec{f}_m
\end{align*} 
diga de qué ecuación s trata y lo que representa. Indique el nombre y 
el significado físico de cada uno de los términos que aparecen en ella.
¿Por qué debe usarse la derivada material en esta ecuación?} 

Se trata de la ecuación de conservación de cantidad de movimiento y representa
la evolución de la cantidad de movimiento de una partícula fluida, 
o la segunda ley de Newton para un fluido.
El primer término se corresponde con un gradiente de presiones que 
resulta siempre normal a la superficie del fluido, el segundo 
término se corresponde con los esfuerzos debidos a la viscosidad
del fluido y el último término es la aportación a la cantidad de 
movimiento de las fuerzas másicas.

Se utiliza la derivada material por que esto nos permite estudiar las
fuerzas que se ejercen sobre una determinada partícula fluida y usar
la descripción lagrangiana en vez de la euleriana.
\subparagraph{Obtenga a partir de esta ecuación la ecuación de Bernouilli
incando las hipótesis que debe realizar. A partir de la ecuación 
anterior encuentre la condición que debe cumplirse en un líquido
en equilibrio indicando las suposicones y condiciones que utilice
para ello. Suponga el caso más general en el que el sistema de 
referencia respecto al que se estudia el fluido no es inercial.
¿Qué ventajas puede implicar este hecho?} 

Consideramos perfecto el fluido, será por lo tanto incompresible
y de viscosidad uniforme.
Por ser incompresible, y de la ley de Navier-Poisson 
\begin{align*}
	\overline{\overline{\tau}}' = 2\mu \overline{\overline{T}}_d
\end{align*}
(Asumiendo ya que el fluido es newtoniano e isótropo) ya que la 
divergencia de la velocidad es nula.

Por ser de viscosidad uniforme verifica que 
$$\nabla \cdot \overline{\overline{\tau}}'= \mu \nabla \times (\nabla \times \vec{v}) $$

Aplicada a la ecuación de balance de cantidad de movimiento 
\begin{align*}
	\rho \frac{D \vec{v}}{Dt} = - \nabla p + \nabla \cdot \overline{\overline{\tau}} + \rho \vec{f}_m = -\nabla p + \mu \nabla \times (\nabla \times \vec{v})  + \rho \vec{f}_m
\end{align*}
Al ser $\rho$ constante podemos escribir 
\begin{align*}
	\frac{D \vec{v}}{Dt} = - \nabla \frac{p}{\rho }+ \vec{f}_m + \nu \nabla \times (\nabla \times \vec{v})  
\end{align*}
Siendo $\nu =  \mu / \rho$.

En el caso de movimientos estacionarios, el coeficiente $\nu$ sea 
despreciable y de que las fuerzas másicas sean conservativas($\vec{f}_m = - \nabla U$ ), entonces 
\begin{align*}
	\frac{D \vec{v}}{Dt}= \nabla \left(\frac{p}{\rho} + U\right)
\end{align*}
Usando la igualdad general 
\begin{align*}
	\frac{D \vec{v}}{D t} = \frac{\partial \vec{v}}{t} + \nabla \left( \frac{\| v\|^2}{2}\right) - \vec{v} \times (\nabla \times \vec{v})
\end{align*}
Como el término del rotacional se anulará al proyectar sobre el 
movimiento, y al considerar movimientos estacionarios el término
de la parcial también, llegamos a la ecuación de Bernouilli 
\begin{align*}
	p + \frac{\rho}{2} \| v \| ^2 + U = C_l (t)
\end{align*}
Si un líquido se encuentra en equilibrio entonces su energía
debida a la traslación será nula. Esto implica que 
\begin{align*}
\nabla \frac{p}{\rho} - \vec{f}_m = 0
\end{align*}
y por tanto la fuerza es necesariamente conservativa, ya que resulta
de un potencial.

La fuerza en un sistema de referencial no inercial es 
\begin{align*}
\vec{f}_m = \vec{a} - \vec{a}_0 - \frac{d}{dt} \vec{\Omega} \times \vec{x} - 2 \vec{\Omega} \times \vec{v} - \vec{\Omega} \times \vec{\Omega} \times \vec{x}
\end{align*}
Y como las únicas fuerzas conservativas que aquí se presentan son todas
menos Coriollis (que es nula) y la debida a la aceleración de la velocidad angular
del sistema, entonces se debe de anular.

Las ventajas de considerar un sistema de referencia no inercial es que
si un sistema de referencia no inercial nos simplifica el problema
podemos además estudiarlo desde un punto de vista estático, no
dinámico.


\end{document}
